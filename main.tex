% -----------------------
% Book of ionic
% -----------------------
\documentclass[a4paper]{book}
\author{ionic online}
\title{ionic}
\usepackage{fontspec,xunicode,xltxtra,makeidx,xecolor}
\usepackage{listings,tabularx,amsfonts,amssymb}
\usepackage{amsmath}
\usepackage{epigraph}
\usepackage{mdwlist}
\usepackage{bchart}
%\usepackage[nonumberlist, toc, acronym]{glossaries}
\usepackage[nonumberlist, toc, acronym]{glossaries}
\usepackage{tikz}
\usetikzlibrary{arrows,shapes,trees,calc,fit,automata,positioning}
\tikzset{
  mybox/.style={
    rectangle,
    rounded corners,
    draw=black
  },
}

\usepackage{color}
\usepackage{titlesec}
\usepackage{amssymb}
\usepackage[colorlinks=false,pdfborder={0 0 0}]{hyperref}
%\setlength{\parindent}{0pt}
%\setlength{\parskip}{\baselineskip}
%\linespread{1.0}

\setcounter{tocdepth}{1}

\setmainfont{WenQuanYi Micro Hei}
\setsansfont{WenQuanYi Micro Hei}
\setmonofont{WenQuanYi Micro Hei}
\setlength{\parindent}{0pt}
\setlength{\parskip}{0.5\baselineskip}
\XeTeXlinebreaklocale "zh"
\XeTeXlinebreakskip=0pt plus 1pt minus 0.1pt

%some global color
\definecolor{light-gray}{rgb}{0.87,0.87,0.87}
\definecolor{light-yellow}{rgb}{0.88,0.92,0.48}
\definecolor{mygreen}{rgb}{0.63,1,0.35}
\definecolor{yanghong}{rgb}{1,0.0,1}

%\newglossary[slg]{symbols}{sym}{sbl}{List of Symbols}

%listing global settings
\lstset{basicstyle=\scriptsize,frame=lines}

%code listing definition
\lstnewenvironment{functionprototype}[1][]
      {\lstset{language=C}\lstset{escapeinside={(*@}{@*)},
       basicstyle=\scriptsize\ttfamily,frame=single,
       numbers=none,numberstyle=\scriptsize,stepnumber=1,numbersep=5pt,
       breaklines=true,
       %firstnumber=last,
           %frame=tblr,
           framesep=5pt,
           showstringspaces=false,
          %identifierstyle=\ttfamily,
           stringstyle=\xecolor{maroon},
        commentstyle=\color{black},
        rulecolor=\color{black},
        xleftmargin=0pt,
        xrightmargin=0pt,
        aboveskip=\medskipamount,
        belowskip=\medskipamount,
               backgroundcolor=\color{white}, #1
}}
{}

%code listing definition
\lstnewenvironment{myccode}[1][]
      {\lstset{language=C}\lstset{escapeinside={(*@}{@*)},
       basicstyle=\scriptsize\ttfamily,
       numbers=left,numberstyle=\scriptsize,stepnumber=1,numbersep=5pt,
       breaklines=true,
       %firstnumber=last,
           %frame=tblr,
           framesep=5pt,
           showstringspaces=false,
           keywordstyle=\itshape\color{blue},
          %identifierstyle=\ttfamily,
           stringstyle=\xecolor{maroon},
        commentstyle=\color{black},
        rulecolor=\color{black},
        xleftmargin=0pt,
        xrightmargin=0pt,
        aboveskip=\medskipamount,
        belowskip=\medskipamount,
               backgroundcolor=\color{white}, #1
}}
{}

\lstnewenvironment{myjavacode}[1][]
      {\lstset{language=JAVA}\lstset{escapeinside={(*@}{@*)},
       breaklines=true,
       %firstnumber=last,
           %frame=tblr,
           framesep=5pt,
           basicstyle=\scriptsize\ttfamily,
           showstringspaces=false,
           keywordstyle=\itshape\color{blue},
          %identifierstyle=\ttfamily,
           stringstyle=\xecolor{maroon},
        commentstyle=\color{black},
        rulecolor=\color{black},
        xleftmargin=0pt,
        xrightmargin=0pt,
        aboveskip=\medskipamount,
        belowskip=\medskipamount,
        backgroundcolor=\color{white}, #1
}}
{}

\lstnewenvironment{myxmlcode}[1][]
      {\lstset{language=XML}\lstset{escapeinside={(*@}{@*)},
       breaklines=true,
       %firstnumber=last,
           %frame=tblr,
           framesep=5pt,
           basicstyle=\scriptsize\ttfamily,
           showstringspaces=false,
           keywordstyle=\itshape\color{blue},
          %identifierstyle=\ttfamily,
           stringstyle=\xecolor{maroon},
        commentstyle=\color{black},
        rulecolor=\color{black},
        xleftmargin=0pt,
        xrightmargin=0pt,
        aboveskip=\medskipamount,
        belowskip=\medskipamount,
        backgroundcolor=\color{white}, #1
}}
{}


%code listing definition
\lstnewenvironment{mymakefilecode}[1][]
      {\lstset{language=c}\lstset{escapeinside={(*@}{@*)},
       basicstyle=\scriptsize\ttfamily,
       numbers=left,numberstyle=\scriptsize,stepnumber=1,numbersep=5pt,
       breaklines=true,
       %firstnumber=last,
           %frame=tblr,
           framesep=5pt,
           showstringspaces=false,
          %identifierstyle=\ttfamily,
           stringstyle=\xecolor{maroon},
        commentstyle=\color{black},
        rulecolor=\color{black},
        xleftmargin=0pt,
        xrightmargin=0pt,
        aboveskip=\medskipamount,
        belowskip=\medskipamount,
               backgroundcolor=\color{white}, #1
}}
{}

\newcommand\vtextvisiblespace[1][.4em]{%
  \mbox{\kern.06em\vrule height.5ex}%
  \vbox{\hrule width#1}%
  \hbox{\vrule height.5ex}}
  
\makeindex

% FIXME , how to use the glossaries?

\input gloss.tex
\makeglossaries

\begin{document}

\maketitle
\tableofcontents
%show or not show the List-of-Figure
%\listoffigures

\part{基础篇}

\input chap_env.tex 

% Here's the Bib
\begin{thebibliography}{99}
\bibitem{bibCRef} Samuel P.Harbison: {\em C - A Reference Manual, Fifth Edition}, C语言参考手册,Printice Hall, Pearson Education Inc., 2002
\bibitem{bibMasterRegExp} Jeffrey E.F.Friedl: {\em Mastering Regular Expressions}, 2012
\bibitem{refDCT1974}N. Ahmed, et al: {\em Discrete Cosine Transform}, DCT转换的论文,IEEE Trans. Computers, 90-93, Jan 1974
\bibitem{refLibjpeg}Independent JPEG Group: {\em www.ijg.org}, libjpeg的官方网址
\bibitem{bibAwkBook} Alfred V.Aho, Brian W.Kernighan, Peter J. Weinberger: {\em The AWK Programming Language}, Addison-Wesley, 1988
\bibitem{bibAPUEPID} W.Richard Stevens: {\em Unix环境高级编程}, Chapter 8:Process Control, Page 187,1993
\bibitem{bibGnuMakefile} GNU Makefile: {\em http://www.gnu.org/software/make}, The GNU Makefile official website
\bibitem{bibVNC} Real VNC: {\em http://www.realvnc.com}, The RealVNC website
\bibitem{bibEffectiveCpp} Scott Meyers: {\em Effective C++, Second Edition}, Pearson Education, 2003
\bibitem{bibEffectiveStl} Scott Meyers: {\em Effective STL}, Pearson Education, 2003
\bibitem{bibAndBook} Yang SongX: {\em Android Book}, 2014
\bibitem{bibLibxml2} LibXML2: {\em http://xmlsoft.org},The LibXML2 package website
\bibitem{bibProtobuf} Google Protobuf: {\em http://code.google.com/p/protobuf/}, The probobuf website
\bibitem{bibLinkLoadLibrary} 俞甲子,石凡,潘爱民: {\em 程序员的自我修养-链接、装载与库}, 2009年
\bibitem{bibSlackWeb} The Slackware Project: {\em http://www.slackware.com}, Slackware official website
\bibitem{bibProtoBufIntro} 刘明: {\em http://www.ibm.com/developerworks/cn/linux/l-cn-gpb/}, Google Protocol Buffer 的使用和原理,2010年
\bibitem{bibCvsBook} David Thomas: {\em 版本控制之道-使用CVS}, 电子工业出版社,2005年
%%Perl
\bibitem{bibMasteringPerl} brian d foy:{\em Mastering Perl}, 精通Perl(影印版),O'Reilly, 2008年
\bibitem{bibLwpBook} Sean M. Burke: {\em Perl \& LWP}, O'Reilly, 1988, 介绍Perl LWP库的参考书
\bibitem{bibCPAN} CPAN: {\em http://www.cpan.org/}, Perl各种功能模块的下载网站
\bibitem{bibAPUEPIDSignal} W.Richard Stevens: {\em Unix环境高级编程}, Chapter 10:Signal, Page 263,1993
\bibitem{bibAndSDK} Android SDK and ADT Bundle: http://developer.android.com/sdk/index.html?hl=sk,Android开发工具下载地址

\bibitem{bibMITAlgorithm}Thomas H.Cormen, Charles E.Leiserson, et.al: {\em Introduction to Algorithms(Third Edition)}, The M.I.T Press, 2009

\bibitem{bibTCPIPVol3}W.Richard Stevens: {\em TCP/IP Illustrated Vol. 3: TCP for Transactions, HTTP, NNTP, and the UNIX Domain Protocols}, 1996
\bibitem{bibDipBook} Kenneth R. Castleman: {\em Digital Image Processing}, Prentice Hall, 1996, 数字图像处理教材
\bibitem{bibZLib} ZLib: {\em http://www.zlib.net/}, 压缩解压库ZLib的官方网站
\bibitem{bibCppJson} json-cpp: {\em http://sourceforge.net/projects/jsoncpp}, C++的JSON数据解析库

\bibitem{bibRabbitMqC} RabbitMQ-C: {\em http://github.com/alanxz/rabbitmq-c}, RabbitMQ的C语言绑定库

\bibitem{bibXFSDK} 讯飞开放平台: {\em http://www.xfyun.cn/} 

%%web framework
\bibitem{bibJavaPlay} Play: {\em http://www.playframework.com/}, Java Play框架的官方网站
\bibitem{bibPerlCatalyst} Catalyst: {\em http://www.catalystframework.org/}, 基于Perl的Web框架
\bibitem{bibDjango} Django: {\em http://www.djangoproject.org/}, 基于Python的Web框架
\bibitem{bibCppCms} CppCMS: {\em http://www.cppcms.com/}, 基于C++的Web框架

\bibitem{bibHadoop} Apache Hadoop: {\em http://hadoop.apache.org/}
\bibitem{bibCordova} Apache Cordova框架: {\em http://cordova.apache.org/}

\bibitem{bivsvdFeature} SVD-Feature, 一个feature-based协同过滤(CF)和排序工具,上海交大Apex实验室开发:{http://www.oschina.net/p/svdfeature}


%%
% begin of 3gpp spec references
\bibitem{bib3gppATCmd} 3gpp TS 27.007: {\em AT command set for User Equipment(UE)}, 手机设备侧的AT命令集标准
\bibitem{bib3gppSIM} 3gpp TS 51.011/11.11: {\em Specification of SIM-ME interface}, SIM卡的3gpp标准
%%
% end of 3gpp spec references
%%

%% begin of RFC references
\bibitem{bibRFC1918} RFC 1918: {\em http://www.rfc-editor.org/rfc/rfc1918.txt}, 私网(Private Internets)地址的分配
\bibitem{bibRFC2083} RFC 2083: {\em http://http://tools.ietf.org/html/rfc2083}, PNG格式标准1.0
\bibitem{bibRFCSIP} RFC 3261: {\em http://www.ietf.org/rfc/rfc3261.txt}, SIP协议的实现
\bibitem{bibRFC3489} RFC 3489: {\em http://www.ietf.org/rfc/rfc3489.txt}, STUN协议的实现
\bibitem{bibRFC3629} RFC 3629: {\em http://www.ietf.org/rfc/rfc3629.txt}, UTF-8编码格式
\bibitem{bibRFC4648} RFC 4648: {\em http://tools.ietf.org/html/rfc4648}, The Base16,Base32,and Base64 Data Encodings
%% end of RFC reference
\end{thebibliography}


% below try show up Index on the Content
\cleardoublepage
\addcontentsline{toc}{chapter}{Index}

\printglossaries

%index
\printindex



%%

\end{document}
